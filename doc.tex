\documentclass[12pt]{article}

\usepackage[margin=1.0in]{geometry}
\usepackage{hyperref}

\title{Assignment \#1}
\author{Jeffrey Klarfeld}

\begin{document}

\maketitle

\section{Approach}
As this assignment was assigned specifically to demonstrate the idea of multithreading an application, a inferior algorithm was specifically chosen.  A sieve based approach was considered (and even partially implemented), but ultimately discarded as it's single threaded run time was too fast to determine any difference in execution time given more resources.
\newline\newline
Ultimately, each order of magnitude from one to eight was assigned a thread in reverse order.  This meant that the thread with the set containing the largest numbers started first ($10^7 \rightarrow 10^8$), and began computing as the remaining threads were still being allocated.
\newline\newline
Each thread would run the method (found in the PrimeFinder class):\newline\newline\texttt{- (void)findPrimesFrom:start upTo:finish;} \newline\newline This method is wrapped in \texttt{@synchronized()} blocks to act as the locking mechanism.  You will see additional locks applied specifically to sections of the method that act upon instance and class variables. These are likely unnecessary, but are included for safety and probably correctness. This method instantiates a temporary mutable array, and ensures that the start point is not a trivially non-prime number.  Then it simply loops over the specified interval skipping all even numbers calling \texttt{- (BOOL)isPrime:number} and adding to the temporary array when a prime is found.  Additionally the number of primes class property is also updated.  The number of primes were kept and counted seperately simply as an academic exercise, it would have been trivial to simply use the count of the \texttt{\_allPrimes} array instead.  Finally, the items from the temporary array are copied into the class array (\texttt{\_allPrimes}) under a lock. 
\newline\newline
Once the thread has finished, it activates a simple global notification to alert the Application Delegate (where the thread allocator resides) that a thread has finished.  Not knowing which thread it was, the count of allocated threads originally established during setup is decremented.  Key-Value-Observing is used to watch the \texttt{threadsAlloced} global variable (all properties declared in appDelegate.h are effectively global due to global availability of the appDelegate instance itself) for changes.  Once the number of allocated threads reaches zero, the timer is stopped and the final output is printed.    


\section{Reasoning On Correctness}
While this algorithm has some flaws, the most glaring of which is the fact that the first four threads will finish almost immediately, it is simple to prove its correctness. It is unimportant what order the threads finish in, just that they've all finished.  There is no shared reads, only shared writes and they are all wrapped in locks. Additionally, there is no requirement that the primes be found in any particular order.  Given this, each thread will be initialized and run completely independent of outside resources until a result has been found.  Only then will an atomic write take place and the thread will return.  Since there is no dependence on outside variables or data to computer results, it can be reasoned that any order of execution is sufficient to solve the problem.

\section{Efficiency}
As previously stated, the algorithm used contained some ineffiencies by choice.  Prevailing of which is the allocation of threads.  This algorithm allocates the problem across all threads evenly, but the work done by each thread is quite unbalanced.  For example: The last thread allocated is given the numbers three through ten, which is a much smaller set to iterate over than the first thread, which is allocated $10^7$ through $10^8$.  There are some precautions which can be taken to attempt to deal with this problem, but with the given allocation of threads there is no situation where all threads can be used for the entire computation.  To improve this situation, the largest jobs are allocated and started first, running on another thread while additional threads are allocated.  Additionally, allocating multiple threads for a single order of magnitude would be simple.  Provided there are 8 threads available, spawning new jobs would just increase the threads counter by 2 instead of one, and would require less than 7 threads allocated to spawn a new job.  Then simply add an additional \texttt{NSDecimalNumber} in between the pre-existing 1 and 10--- Which would be 5 in this case. Then each is allocated with as $1 \rightarrow 5, 5 \rightarrow 10$.
\newline\newline
The only other computationally expensive method is \texttt{isPrime:number}. The simple approach to determining primality is to start with an odd number and check if its modulus with n, where n is a loop counter running from 3 to $n/2$, is zero. This method is responsible for huge run times after $10^7$, so it was replaced with an alternate method that runs in less than $O(N)$ time. 

\section{Experimental Evaluation} 
Starting from purposely enacted obsolescence, the algorithm was never going to be faster than using a sieve.  Additionally, due to the allocation of threads, the algorithm doesn't see much more than a speedup of two. However, it uses small amounts of memory, and operates individual threads with an effective $O(n^2)$ algorithm that does not deadlock or starve. With this in mind, the application can be deemed a success, albeit a somewhat limited one.  

\end{document}